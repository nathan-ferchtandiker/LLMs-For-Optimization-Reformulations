\documentclass{article}
\usepackage{amsmath} % For advanced math formatting
\usepackage{enumitem} % For better itemize control

\begin{document}

\section{Sets}
\begin{itemize}[leftmargin=*,nosep]
    \item $P$: set of all simple paths from supplier to beneficiary camps.
    \item $K$: set of commodities.
    \item $L$: set of nutrients.
    \item $N_B$: set of beneficiary camps.
\end{itemize}

\section{Parameters}
\begin{itemize}[leftmargin=*,nosep]
    \item $c_{pk}$: cost of shipping one kg of commodity $k \in K$ along path $p \in P$.
    \item $q_k$: procurement cost per kg of commodity $k \in K$.
    \item $\mathrm{nutval}_{k\ell}$: nutrient-$\ell$ content (per kg) of commodity $k \in K$.
    \item $\mathrm{nutreq}_\ell$: per-person requirement for nutrient $\ell \in L$.
    \item $\mathrm{dem}_j$: number of beneficiaries at camp $j \in N_B$.
    \item $e_{jp} = 
    \begin{cases}
        1, & \text{if path } p \text{ ends at camp } j, \\
        0, & \text{otherwise}
    \end{cases}$\quad for all $j \in N_B$, $p \in P$.
\end{itemize}

\section{Variables}
\begin{itemize}[leftmargin=*,nosep]
    \item $x_{pk} \geq 0$: amount (kg) of commodity $k \in K$ shipped along path $p \in P$.
    \item $R_k \geq 0$: ration size (kg per person) of commodity $k \in K$.
\end{itemize}

\section{The Model}
\begin{align*}
    \min \quad & \sum_{p \in P} \sum_{k \in K} c_{pk} \; x_{pk} + \sum_{k \in K} q_k \cdot \left( \sum_{p \in P} x_{pk} \right) \\
    & \text{Minimize total cost of procurement and transportation} \\[6pt]
    \text{s.t.} \quad & \sum_{p \in P} e_{jp} \, x_{pk} \geq \mathrm{dem}_j \cdot R_k, \quad \forall j \in N_B,\; \forall k \in K \\
    & \text{Ensure each camp receives required ration quantities} \\[6pt]
    & \sum_{k \in K} \mathrm{nutval}_{k\ell} \cdot R_k \geq \mathrm{nutreq}_\ell, \quad \forall \ell \in L \\
    & \text{Ensure minimum nutritional content per person} \\[6pt]
    & x_{pk} \geq 0, \quad \forall p \in P, \forall k \in K \\
    & R_k \geq 0, \quad \forall k \in K \\
    & \text{Define variable domains}
\end{align*}

\end{document}